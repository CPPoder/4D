\documentclass[titlepage]{scrartcl}
% Code Darstellung
\usepackage{listings}
\usepackage{listingsutf8}
\usepackage{multicol}

%lange Tabellen
\usepackage{longtable}
%Referenzen zwischen unterschiedlichen Dateien
\usepackage{xr}
\externaldocument{theorie}
\usepackage{lscape}
%Deutsche Sprachunterstützung
\usepackage[utf8]{inputenc}
%\usepackage[ngerman]{babel}
\usepackage[english]{babel}
\usepackage{marvosym}
\DeclareUnicodeCharacter{20AC}{\EUR}

%Für das Einbinden von Bildern
\usepackage{graphicx}

%Für das Fixieren von Bildern
\usepackage{float}

%Tabellen
\usepackage{array}

%Tabellen automatisch schoener
\usepackage{booktabs}

%Caption
\usepackage{caption}
\usepackage{subcaption}

%Formeln
\usepackage{mathtools}
\usepackage{amsmath}
\usepackage{amssymb}
\usepackage{amstext}
\usepackage{dsfont}

%\usepackage{mnsymbol}

% Interssante natbib Optionen: 
% numbers : Nummerierte Zitateinheiten
% sort&compress : Bei mehrfachen Zitaten, Sortierung und ggf. Verkürzungen
%\usepackage[]{natbib}

%Vectorpfeile schöner
\usepackage{esvect}

%Formatierung
\usepackage[T1]{fontenc}
\usepackage{lmodern}
\usepackage{microtype}

%Schaltbilder malen
%\usepackage[europeanresistors,cuteinductors,siunitx]{circuitikz}
\usepackage{comment}
\usepackage{csquotes}

%Formatierungsanweisungen
\newcommand{\wichtig}[1]{\underline{\large{#1}}}
%\newcommand{\aref}[1]{Abb. \ref{#1}}
\newcommand{\aref}[1]{figure \ref{#1}}
\newcommand{\R}{\mathbb{R}}
\newcommand{\K}{\mathbb{K}}
\newcommand{\C}{\mathbb{C}}

%Klickbare Referenzen
\usepackage[hidelinks]{hyperref}

%Kein Einrücken
\setlength{\parindent}{0pt}

%Breite Tabellen
\usepackage{tabularx}

\begin{document}

\title{Documentation of 4D engine}

\maketitle

\section{Structure of the project}

List of classes: 

\begin{itemize}
\item Framework: 

Contains opening the renderWindow

\item World

Several tasks: hosting all objects and the observer, 

\item Element:

Basic class of all classes which have the functions render, update, handleEvents

\item Object:

Basic class of all objects in space

\item Cuboid:

Describes a cuboid in space

\item Observer:

Describes the position and orientation of the observer in space. Position is encoded in a 4D vector, orientation (\enquote{mView}) is encoded in a 4D matrix (in SO(4)) describing the transformation from stanard coordinates to the observer coordinates 

\end{itemize}

\section{Idea of the 4D projection and controlling}

Observer has the variable fd::Vector4f mView which describes the transformation from observer coordinates ($S'$) to standard coordinates ($S$):

\begin{equation}
mView: S' \to S
\end{equation}

A change of coordinates can be easily expressed in terms of the current observer coordintes as a trafo R (in SO(4)). The new observers coordinates are then given by $S'_{new} = R \cdot S'_{old}$ and the new mView matrix then is described by 

\begin{equation}
mView_{new} = mView_{old} \cdot R^{-1},
\end{equation}

as then $mView : S'_{new} \to S'_{old} \to S$.

The advantage of this convention is that during the calculation of a points components in observers coordinates from its standard coordinates, one has to use the map $S' \to S$ as it's the inverse the map mapping the system $S$ on $S'$. Hence one does not need to invert the mView.

The total transformation of a points coordinates in S to its components in S' is then given by first applying the translation (which is given in terms of S) and then applying the map mView:

\begin{equation}
\vec{x}_{S'} = mView \cdot (\vec{x}_S - \vec{mPosition}).
\end{equation}

The standard coordinates of S shall be expressed by $x_1, x_2, x_3, x_4$. For some reason, I would like to introduce new coordinates for the initial state of the observer: 

\begin{equation}
y = (0, 0, \frac{1}{\sqrt{2}}, -\frac{1}{\sqrt{2}})^T
\end{equation} and

\begin{equation}
v = (0, 0, \frac{1}{\sqrt{2}}, \frac{1}{\sqrt{2}})^T
\end{equation}

and ($x_1, x_2, y, v$) the initial observers coordinates. 

Moving of the observer then is implemented by: 

1) Moving in direction v (\enquote{view}) with the current speed with the possibility of accelerating and slowing down. 

2) Moving the view in either positive or negative $x_1, x_2$ or $y$ direction with (constant) angular speed. 

The rotations describing the three possibilities in 2) are given by 4x4 matrices: 

\begin{equation}
R_{x_1}(\phi) = \begin{array}{cccc}
\cos(\phi) & 0 & 0 & \sin(\phi) \\ 
0 & 1 & 0 & 0 \\ 
0 & 0 & 1 & 0 \\ 
-\sin(\phi) & 0 & 0 & \cos(\phi)
\end{array} 
\end{equation}

\begin{equation}
R_{x_2}(\phi) = \begin{array}{cccc}
1 & 0 & 0 & 0 \\ 
0 & \cos(\phi) & 0 & \sin(\phi) \\ 
0 & 0 & 1 & 0 \\ 
0 & -\sin(\phi) & 0 & \cos(\phi)
\end{array} 
\end{equation}

\begin{equation}
R_{y}(\phi) = \begin{array}{cccc}
1 & 0 & 0 & 0\\ 
0 & 1 & 0 & 0 \\ 
0 & 0 & \cos(\phi) & \sin(\phi) \\ 
0 & 0 & -\sin(\phi) & \cos(\phi)
\end{array} 
\end{equation}.

Controlling is done by using the keys W, A, S, D, X and Y for rotation the direction in which the observer looks and using Up and Down (right hand) for moving in direction of v forwards or backwards. 



\end{document}