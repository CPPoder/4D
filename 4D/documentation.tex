\documentclass[titlepage]{scrartcl}
% Code Darstellung
\usepackage{listings}
\usepackage{listingsutf8}
\usepackage{multicol}

%lange Tabellen
\usepackage{longtable}
%Referenzen zwischen unterschiedlichen Dateien
\usepackage{xr}
\externaldocument{theorie}
\usepackage{lscape}
%Deutsche Sprachunterstützung
\usepackage[utf8]{inputenc}
%\usepackage[ngerman]{babel}
\usepackage[english]{babel}
\usepackage{marvosym}
\DeclareUnicodeCharacter{20AC}{\EUR}

%Für das Einbinden von Bildern
\usepackage{graphicx}

%Für das Fixieren von Bildern
\usepackage{float}

%Tabellen
\usepackage{array}

%Tabellen automatisch schoener
\usepackage{booktabs}

%Caption
\usepackage{caption}
\usepackage{subcaption}

%Formeln
\usepackage{mathtools}
\usepackage{amsmath}
\usepackage{amssymb}
\usepackage{amstext}
\usepackage{dsfont}

%\usepackage{mnsymbol}

% Interssante natbib Optionen: 
% numbers : Nummerierte Zitateinheiten
% sort&compress : Bei mehrfachen Zitaten, Sortierung und ggf. Verkürzungen
%\usepackage[]{natbib}

%Vectorpfeile schöner
\usepackage{esvect}

%Formatierung
\usepackage[T1]{fontenc}
\usepackage{lmodern}
\usepackage{microtype}

%Schaltbilder malen
%\usepackage[europeanresistors,cuteinductors,siunitx]{circuitikz}
\usepackage{comment}
\usepackage{csquotes}

%Formatierungsanweisungen
\newcommand{\wichtig}[1]{\underline{\large{#1}}}
%\newcommand{\aref}[1]{Abb. \ref{#1}}
\newcommand{\aref}[1]{figure \ref{#1}}
\newcommand{\R}{\mathbb{R}}
\newcommand{\K}{\mathbb{K}}
\newcommand{\C}{\mathbb{C}}

%Klickbare Referenzen
\usepackage[hidelinks]{hyperref}

%Kein Einrücken
\setlength{\parindent}{0pt}

%Breite Tabellen
\usepackage{tabularx}

\begin{document}

\title{Documentation of 4D engine}

\maketitle


\section{Controlling}
Controlling is done by using the keys W, A, S, D, X and Y for rotation the direction in which the observer looks and using Up and Down (right hand) for moving in direction of v (explained later) forwards or backwards. 


\section{Structure of the project}

List of classes: 

\begin{itemize}
\item Framework: 

Contains opening the renderWindow

\item World

Several tasks: hosting all objects and the observer, 

\item Element:

Basic class of all classes which have the functions render, update, handleEvents

\item Object:

Basic class of all objects in space

\item Cuboid:

Describes a cuboid in space

\item Observer:

Describes the position and orientation of the observer in space. Position is encoded in a 4D vector, orientation (\enquote{mView}) is encoded in a 4D matrix (in SO(4)) describing the transformation from stanard coordinates to the observer coordinates 

\end{itemize}

\section{Idea of the 4D projection and controlling and their visibilities}


\subsection{Parallel Projection}
\subsubsection{Projecting}
The idea of parallel projection is that any line in $x_3$ or $x_4$ direction is drawn in a certain direction (here: $x_3$ in a 45 degree angle to the upper right, $x_4$ to the upper left) and distances in one of those directions is scaled by a certain factor only. This means e.g. that objects in far distance have the same size as close objects. 

The projection $ p : \mathbb{R}^4 \to \mathbb{R}^2$ was implemented by the following matrix: 
\newline

$\begin{array}{cccc}
1 & 0 & .5 & -.5 \\ 
0 & 1 & -.5 & -.5
\end{array} $

\subsubsection{Visibility of elements}

As the projection performed goes from 4D to 2D, there are two axes orthogonal to the plane projected onto. 

It is assumed, that a point is visible, if its coordinates in both $x_3$ and $x_4$ are positive. (The situation can be visualized by using a 2D coordinate system - a point is visible iaof it lies in the I. quadrant.)

To correctly handle an edge which is bounded by two points, one has to use an algorithm to find the visible part of the edge. 

A) Both points are visible --> the whole edge is visible as the visible part is convex

B) One point is visible ($\vec{x}$), the other one not ($\vec{y}$)$\rightarrow$
Define $t^1 = \frac{x^1}{x^1 - y^1}, t^2 = \frac{x^2}{x^2 - y^2}$ and set $ t = \min(t^1, t^2), \vec{z} = (1 - t) \vec{x} + t \vec{y}$.

--> The new edge is given by $[\vec{x}, \vec{z}]$.

C) Both points are invisible
1) $\vec{x}, \vec{y}$ lie in the II. / IV. quadrant disjointly (each one of them in another quadrant)

Define $t^1 = \frac{x^1}{x^1 - y^1}, t^2 = \frac{x^2}{x^2 - y^2}$. If $x^1 (t^1 - t^2) > 0 \rightarrow $ the intervall $[(1 - t^1) \vec{x} + t^1 \vec{y}, (1 - t^2) \vec{x} + t^2 \vec{y}]$ is visible, else nothing is visible. 

\subsection{Spatial projection}
The spatial projection implemented uses the idea of all parallel lines \enquote{focusing} on a point lying in infty distance to the observer. 

Such a projection can be achieved by using the projection map $p : \mathbb{R}^3 \to \mathbb{R}^2$ defined by: 

\begin{equation}
\vec{p}(x_1, x_2, x_3, x_4) = \vec{\alpha} \cdot [\vec{f_1} + \vec{f_2} - (\vec{f_1} - \vec{x_1x_2}) (x_3 - z_0 + 1)^{-1} - (\vec{f_2} - \vec{x_1x_2}) (x_4 - z_0 + 1)^{-1}]
\end{equation}
where $\vec{\alpha}$ is a vector controlling the scaling (the zoom), $\vec{x_1 x_2}$ denotes the vector defined by $(x1, x2)^T$, $\vec{f_1}, \vec{f_2}$ the focus points to which all lines in $x_3$ or $x_4$ direction head and $z_0$ a parameter giving the smallest distance which can be focused on by the observer. 
The reason here is, that the physical eye can observe objects starting from a certain distance as the lense of the eye can only be bended in a certain range. Furthermore, things being at distance zero would have the size infinity. 
This formula is inspired by the physics of the eye. 
Additionally coloring is being used. \\

Warning: this projection seems to be quite silly due to several reasons: 

- if a point has $x_3 >> 0$ but $x_4$ in the range of 0, than a small change in $x_4$ has a large impact the projected point. 

The notion of visibility has to be discussed! Probably according to pure color!

\subsection{The pure-color projection}

!!Warning: In this mode, restrictedVisibility always has to be true. 

\subsubsection{Projecting}
This projection uses a standard 3D projection and encodes the fourth dimension in color only. Its implementation is quite simple, will have to be compared to the spatial 4D one. 

One gets the projection:
\begin{equation}
p(x_1, x_2, x_3, x_4) =  \vec{x_1 x_2} \frac{z_0}{x_3}
\end{equation}
where $z_0$ again is the parameter described above. Note that objects having $z_0 - 1 < x_3 < z_0$ are bigger with size going to infinity for the left inequality approaching equality and that for $z_0 - 1 > x_3$ the formula is nonsense as its outcome is negative at a certain point. 

\subsubsection{The visibility}

Following from the above construction of the projecting method, it seems senseful to call a point visible if it has $x_3 > z_0$ and invisible else. 

For the fourth dimension it is questionable whether their should be an invisible part. At the moment, I will implement that a point is only visible if he also has $x_4 > z_0$. 

\section{Saving the motion of the observer}

Observer has the variable fd::Vector4f mView which describes the transformation from observer coordinates ($S'$) to standard coordinates ($S$):

\begin{equation}
mView: S' \to S
\end{equation}

A change of coordinates can be easily expressed in terms of the current observer coordintes as a trafo R (in SO(4)). The new observers coordinates are then given by $S'_{new} = R \cdot S'_{old}$ and the new mView matrix then is described by 

\begin{equation}
mView_{new} = R \cdot mView_{old} 
\end{equation}

for a reasoning which is a bit confusing for me right now. It is important to look at the system, in which the rotation is done. This system is fixed!git 

The advantage of this convention is that during the calculation of a points components in observers coordinates from its standard coordinates, one has to use the map $S' \to S$ as it's the inverse the map mapping the system $S$ on $S'$. Hence one does not need to invert the mView.

The total transformation of a points coordinates in S to its components in S' is then given by first applying the translation (which is given in terms of S) and then applying the map mView:

\begin{equation}
\vec{x}_{S'} = mView \cdot (\vec{x}_S - \vec{mPosition}).
\end{equation}

The standard coordinates of S shall be expressed by $x_1, x_2, x_3, x_4$. For some reason, I would like to introduce new coordinates for the initial state of the observer: 

\begin{equation}
y = (0, 0, \frac{1}{\sqrt{2}}, -\frac{1}{\sqrt{2}})^T
\end{equation} and

\begin{equation}
v = (0, 0, \frac{1}{\sqrt{2}}, \frac{1}{\sqrt{2}})^T
\end{equation}

and ($x_1, x_2, y, v$) the initial observers coordinates. 

Moving of the observer then is implemented by: 

1) Moving in direction v (\enquote{view}) with the current speed with the possibility of accelerating and slowing down. 

2) Moving the view in either positive or negative $x_1, x_2$ or $y$ direction with (constant) angular speed. 

The rotations describing the three possibilities in 2) are given by 4x4 matrices: 

\begin{equation}
R_{x_1}(\phi) = \begin{array}{cccc}
\cos(\phi) & 0 & 0 & \sin(\phi) \\ 
0 & 1 & 0 & 0 \\ 
0 & 0 & 1 & 0 \\ 
-\sin(\phi) & 0 & 0 & \cos(\phi)
\end{array} 
\end{equation}

\begin{equation}
R_{x_2}(\phi) = \begin{array}{cccc}
1 & 0 & 0 & 0 \\ 
0 & \cos(\phi) & 0 & \sin(\phi) \\ 
0 & 0 & 1 & 0 \\ 
0 & -\sin(\phi) & 0 & \cos(\phi)
\end{array} 
\end{equation}

\begin{equation}
R_{y}(\phi) = \begin{array}{cccc}
1 & 0 & 0 & 0\\ 
0 & 1 & 0 & 0 \\ 
0 & 0 & \cos(\phi) & \sin(\phi) \\ 
0 & 0 & -\sin(\phi) & \cos(\phi)
\end{array} 
\end{equation}.







\end{document}